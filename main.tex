\documentclass{article}
\usepackage{fontspec}     
\usepackage[UTF8]{ctex}
\setCJKmainfont{FandolSong-Regular}
\usepackage{amsmath}     
\usepackage{amssymb}     
\usepackage{enumitem}    
\usepackage{needspace}
\usepackage{listings}
\usepackage{booktabs}
\usepackage{float}
\usepackage{geometry}
\geometry{
    left=3cm,      % 左边距
    right=3cm,     % 右边距
    top=2.5cm,     % 上边距
    bottom=2.5cm,  % 下边距
    headheight=15pt, % 页眉高度
    includehead,   % 包含页眉
    includefoot    % 包含页脚
}
\title{编程实训作业}
\author{陆博福}
\date{2025年 9月}

\lstset{
    basicstyle=\ttfamily,        % 等宽字体
    keywordstyle=\bfseries,      % 关键字加粗
    commentstyle=\itshape,       % 注释斜体
    stringstyle=\ttfamily,       % 字符串用等宽
    frame=single,                % 加边框
    breaklines=true,             % 自动换行
    numbers=left,                % 显示行号
    numberstyle=\tiny,           % 行号字体小
}

\begin{document}

\maketitle

\section{查漏补缺}
以下是一些在课堂学习过程中提及的、较为陌生或不熟悉的词汇,在进一步查阅资料学习后记录了笔记。
\subsection{最小点覆盖与最大边匹配}

边匹配是一个边集合 $ M \subseteq E $,其中任意两条边不共享同一个顶点。
即:对于任意 $ e_1, e_2 \in M $,$ e_1 \neq e_2 $,且 $ e_1 $ 和 $ e_2 $ 没有公共顶点。

点覆盖是一个顶点集合 $ C \subseteq V $,使得每一条边 $ (u, v) \in E $ 至少有一个端点属于 $ C $。
即:对任意边 $ (u, v) \in E $,有 $ u \in C $ 或 $ v \in C $。

在一个二分图中,最大匹配的大小等于最小点覆盖的大小。
假设图 $ G = (V, E) $ 是一个二分图,其中
$ M $ 是一个边匹配,
$ C $ 是一个点覆盖。

则:
最大匹配的大小为 $ |M_{\max}| $,最小点覆盖的大小为 $ |C_{\min}| $

根据 König 定理:
$$
|M_{\max}| = |C_{\min}|
$$

应用场景:任务分配、 网络流中的最大流与最小割、计算机视觉中的图像分割、电路布线与逻辑门设计、生物信息学中的基因调控网络、 社交网络中的影响力传播、在线广告投放策略等。

\needspace{5cm} 
\subsection{匈牙利算法}

匈牙利算法(Hungarian Algorithm)是一种用于解决分配问题(Assignment Problem)的算法。它主要用于在一个成本矩阵中找到一种最优的分配方式,使得总成本最小(或最大)。例如,在一个任务分配问题中,每个工人只能分配一个任务,每个任务也只能由一个工人完成,而目标是使总成本最低。

算法的核心是通过一系列操作来寻找\textbf{最小权完美匹配},其步骤大致如下:

\textbf{构造代价矩阵}:将问题转化为一个矩阵,其中元素表示某人做某事的成本或时间。

\textbf{行减法和列减法}:对每一行和每一列减去该行或该列的最小值,使得至少有一行或一列出现0。

\textbf{尝试覆盖所有0}:用最少的直线覆盖所有0,判断是否可以找到一个完美匹配。

\textbf{调整矩阵}:如果不能找到完美匹配,则调整矩阵,重复上述过程,直到找到一个完美匹配为止。

\begin{lstlisting}[caption={匈牙利算法伪代码}, label=code:hungarian]
for each row:
    subtract min(row)
for each column:
    subtract min(column)
while lines < n:
    cover zeros with minimum lines
    if lines < n:
        find min uncovered, adjust matrix
find optimal assignment
\end{lstlisting}

\needspace{5cm}
\subsection{阿列夫零}

阿列夫零(Aleph-null,符号为$\aleph_0$)是数学中用来表示最小的无限基数的符号,它代表的是可数无限集合的大小。例如,自然数集合 \{0, 1, 2, 3, ...\} 的基数就是阿列夫零。

在集合论中,阿列夫零是第一个阿列夫数,用于描述无限集合的“大小”。与之相对的是更大的无限基数,如阿列夫一($\aleph_1$)、阿列夫二($\aleph_2$)等,它们分别对应不可数无限集合的大小。

简单来说,任何与自然数集合具有相同基数的无限集合都是可数无限的,而像实数集合这样的无限集合则是不可数无限的,其基数大于阿列夫零。

\subsection{共轭先验}
共轭先验(Conjugate Prior)是贝叶斯统计中的一个重要概念,用于简化后验分布的计算。它的基本思想是:\textbf{选择一个与似然函数具有相同形式的先验分布},这样在计算后验分布时,可以保持相同的分布形式,从而避免复杂的积分运算。

如果一个先验分布 $ p(\theta) $ 与似然函数 $ p(x|\theta) $ 的乘积(即联合分布)在归一化之后,仍然是一个已知的、容易处理的分布形式,那么这个先验分布就被称为\textbf{共轭先验}。

换句话说,若:
$$
p(\theta | x) \propto p(x|\theta) \cdot p(\theta)
$$
且 $ p(\theta | x) $ 是某种已知分布(如正态分布、贝塔分布等),那么 $ p(\theta) $ 就是 $ p(x|\theta) $ 的共轭先验。
\begin{description}
    \item[计算简便] 共轭先验与似然函数属于同一概率分布族,因此后验分布的形式与先验分布相同。这使得后验分布的计算变得简单,无需复杂的数值积分或近似方法。
    \item[易于更新] 当获得新的数据时,只需对先验参数进行更新,而不需要重新计算整个后验分布。这种特性使得共轭先验非常适合在线学习和动态模型更新。
    \item[适合解析求解] 在许多情况下,共轭先验允许直接解析地得到后验分布,而不是依赖于蒙特卡洛模拟或其他数值方法。
\end{description}

\begin{table}[H]
    \centering
    \begin{tabular}{ccc}
        \toprule
        似然函数 & 共轭先验 & 后验分布 \\
        \midrule
        $\mathcal{N}(\mu)$ & $\mathcal{N}(\mu_0, \sigma^2_0)$ & $\mathcal{N}(\mu_n, \sigma^2_n)$ \\
        $\mathcal{N}(\sigma^2)$ & $\text{Inv-Gamma}(\alpha_0, \beta_0)$ & $\text{Inv-Gamma}(\alpha_n, \beta_n)$ \\
        $\text{Bernoulli}(p)$ & $\text{Beta}(\alpha_0, \beta_0)$ & $\text{Beta}(\alpha_n, \beta_n)$ \\
        $\text{Binomial}(p)$ & $\text{Beta}(\alpha_0, \beta_0)$ & $\text{Beta}(\alpha_n, \beta_n)$ \\
        $\text{Poisson}(\lambda)$ & $\text{Gamma}(\alpha_0, \beta_0)$ & $\text{Gamma}(\alpha_n, \beta_n)$ \\
        $\text{Gamma}(\alpha)$ & $\text{Inv-Gamma}(\alpha_0, \beta_0)$ & $\text{Inv-Gamma}(\alpha_n, \beta_n)$ \\
        \bottomrule
    \end{tabular}
    \caption{常见分布及其共轭先验与后验分布}
    \label{tab:conjugate_priors}
\end{table}



\section{二项分布与泊松二项复合分布的比较分析}
现有两种模型,一是二项分布,固定投100次硬币,三次实验正面朝上的次数分别为49、50、51次,二是先泊松分布再二项分布,投硬币固定计时2分钟,平均投100次,实验得到的正面朝上次数也是49、50、51。下面比较两种模型的差异:
\subsection{两种模型设定及其期望与方差}
    \subsubsection{单独二项分布}  $ X \sim B(n, p) $ $ X $ 表示在 $ n $ 次独立重复试验中成功次数,每次成功的概率为 $ p $。
    \begin{itemize}
        \item 期望:$E(X) = np$
        \item 方差:$Var(X) = np(1-p)$
    \end{itemize}
    \subsubsection{先泊松后二项}$ X \sim B(N, p),
    N \sim Poisson(\lambda)$ 设$ N $ 表示在可能进行试验的次数,$X$表示$N$次试验成功的次数,每次成功的概率为 $ p $
    \begin{itemize}
        \item 边缘期望: $$
  E(X) = E(E(X|N)) = E(Np) = pE(N) = \lambda p
  $$
  即 $ E(X) = \lambda p $
        \item 边缘方差:$$
  Var(X) = E(Var(X|N)) + Var(E(X|N))
  $$
  其中:
  $ E(Var(X|N)) = E(Np(1-p)) = p(1-p)E(N) = p(1-p)\lambda $
  $ Var(E(X|N)) = Var(Np) = p^2 Var(N) = p^2 \lambda $
  
  因此:
  $$
  Var(X) = p(1-p)\lambda + p^2 \lambda = \lambda p
  $$
    \end{itemize} 

即
\begin{table}[htbp]
\centering
\caption{两种模型设定及其期望与方差}
\begin{tabular}{|c|c|c|c|}
\hline
\textbf{模型设定} & \textbf{分布} & \textbf{期望 $ E(X) $} & \textbf{方差 $ \text{Var}(X) $} \\
\hline
单独二项分布 & $ X \sim B(n, p) $ & $ np $ & $ np(1-p) $ \\
\hline
先泊松后二项 & $ X \mid N \sim B(N, p), N \sim \text{Poisson}(\lambda) $ & $ \lambda p $ & $ \lambda p $ \\
\hline
\end{tabular}
\end{table}

\subsection{矩估计}
矩估计的核心思想:用样本的矩(如均值、方差等)去估计总体的矩。
\subsubsection{二项分布}

总体期望(一阶矩):$ E(X_1) = np_1 $

总体方差(二阶矩):$ \text{Var}(X_1) = np_1(1-p_1) $

\paragraph{计算样本均值}
$$
\bar{X_1} = \frac{1}{n} \sum_{i=1}^{n} X_1i
$$

\paragraph{用样本均值估计总体均值}
$$
E(X_1) = np_1 = \bar{X_1}
$$

解得:

$$
\hat{p_1} = \frac{\bar{X_1}}{n}
$$

\subsubsection{先泊松再二项}
\paragraph{计算样本均值}
$$
E(X_2) = \sum_{n=0}^\infty \sum_{x=0}^n x \cdot f_{X_2|N_2}(x | n) \cdot f_{N_2}(n)
$$

其中:

$ f_{X_2|N_2}(x | n) = \binom{n}{x} p_2^x (1-p_2)^{n - x} $ ($ x \leq n $)

$ f_{N_2}(n) = \frac{\lambda^n e^{-\lambda}}{n!} $

因此:

$$
E(X_2) = \sum_{n=0}^\infty E(X_2 \mid N_2 = n) \cdot f_{N_2}(n) = \sum_{n=0}^\infty np_2 \cdot f_{N_2}(n)
$$

$$
= p_2 \sum_{n=0}^\infty n f_{N_2}(n) = p_2 \cdot E(N_2)
$$

因为 $ N_2 \sim \text{Poisson}(\lambda) $,所以:

$$
E(N_2) = \lambda
$$

因此:

$$
E(X_2) = p_2 \cdot \lambda
$$

\paragraph{用样本均值估计总体均值}
$$
\bar{X_2} = p_2 \cdot \lambda
$$

解得:

$$
\hat{p_2} = \frac{\bar{X_2}}{\lambda}
$$




\subsection{矩估计的无偏性与有效性}
\subsubsection{单独二项分布}   
\paragraph{无偏性}

$$
E[\hat{p}_1] = E\left[ \frac{x_1}{n} \right] = \frac{1}{n} E[x_1] = \frac{1}{n} (np_1) = p_1
$$
——无偏(期望等于自身)

\paragraph{有效性}

$$
\text{Var}(\hat{p}_1) = \text{Var}\left( \frac{x_1}{n} \right) = \frac{1}{n^2} \cdot \text{Var}(x_1)
$$
由于$ \text{Var}(X_1) = np_1(1 - p_1) $因此:

$$
\text{Var}(\hat{p}_1) = \frac{p_1(1 - p_1)}{n}
$$
Fisher 信息量:
$$
I(p_1) = \frac{n}{p_1(1 - p_1)}
$$

Cramér-Rao 下界:
$$
\frac{1}{n I(p_1)} = \frac{p_1(1 - p_1)}{n}
$$
——有效(在 Cramér-Rao 下界下)

\subsubsection{先泊松再二项}
\paragraph{无偏性}
$$
E[\hat{p}_2] = E\left[ \frac{x_2}{\lambda} \right] = \frac{1}{\lambda} E[x_2] = \frac{1}{\lambda} (p_2 \lambda) = p_2
$$
——无偏(期望等于自身)

\paragraph{有效性}
$$
\text{Var}(\hat{p}_2) = \frac{1}{\lambda^2} \cdot \text{Var}(x_2)
$$

因为 $ \text{Var}(x_2) = p_2 \lambda $,所以:

$$
\text{Var}(\hat{p}_2) = \frac{p_2 \lambda}{\lambda^2} = \frac{p_2}{\lambda}
$$
方差比直接对二项分布估计要大$\frac{n}{(1-p)\lambda}$倍(若假设$\lambda=n$则为$\frac{1}{1-p}$),但仍是有效的
    

\subsection{最大似然估计}
最大似然估计的核心思想:找到一组参数值,使得在这组参数下,观测到当前数据的概率(即似然函数)最大。
\paragraph{数学表达}

假设我们有一组独立同分布(i.i.d.)的观测数据 $ x_1, x_2, \dots, x_n $,这些数据来自一个概率分布,该分布的概率密度函数(PDF)或概率质量函数(PMF)为 $ f(x \mid \theta) $,其中 $ \theta $ 是我们要估计的未知参数。

似然函数定义为:

$$
L(\theta) = \prod_{i=1}^{n} f(x_i \mid \theta)
$$

为了方便计算,通常取对数似然函数:

$$
\ell(\theta) = \log L(\theta) = \sum_{i=1}^{n} \log f(x_i \mid \theta)
$$

最大似然估计的目标是找到使对数似然函数最大的参数值 $ \hat{\theta} $:

$$
\hat{\theta}_{\text{MLE}} = \arg\max_{\theta} \ell(\theta)
$$


\subsubsection{二项分布}
\paragraph{概率质量函数}
$$
P(X_1 = x) = \binom{n}{x} p_1^x (1 - p_1)^{n - x}, \quad x = 0, 1, \dots, n
$$
\paragraph{对数似然函数}
假设观测数据为 $ x_1 $,则似然函数为:

$$
L(p_1) = \binom{n}{x_1} p_1^{x_1} (1 - p_1)^{n - x_1}
$$

对数似然函数:

$$
\ell(p_1) = \log L(p_1) = \log \binom{n}{x_1} + x_1 \log p_1 + (n - x_1) \log(1 - p_1)
$$

\paragraph{最大化对数似然函数}
对 $ p_1 $ 求导并令其为零:

$$
\frac{d\ell}{dp_1} = \frac{x_1}{p_1} - \frac{n - x_1}{1 - p_1} = 0
$$

解得:

$$
\frac{x_1}{p_1} = \frac{n - x_1}{1 - p_1} \Rightarrow x_1(1 - p_1) = p_1(n - x_1)
$$

$$
x_1 - x_1 p_1 = n p_1 - x_1 p_1 \Rightarrow x_1 = n p_1 \Rightarrow p_1 = \frac{x_1}{n}
$$
最大似然估计量为:
$$
\hat{p}_1 = \frac{x_1}{n}
$$
\subsubsection{先泊松再二项}
\paragraph{联合分布与边缘分布}
\begin{itemize}
    \item 联合分布
    \texttt{$$P(X_2 = x, N = n) = P(N = n) \cdot P(X_2 = x | N = n)$$}
    \texttt{$$= e^{-\lambda} \frac{\lambda^n}{n!} \binom{n}{x} p_2^x (1 - p_2)^{n - x}$$}
    \item 边缘分布
    $$
    P(X_2 = x) = \sum_{n=x}^\infty e^{-\lambda} \frac{\lambda^n}{n!} \binom{n}{x} p_2^x (1 - p_2)^{n - x}
    $$
    这个求和可以简化为:
    \texttt{$$
P(X_2 = x) = e^{-\lambda} \frac{(\lambda p_2)^x}{x!} \sum_{n=x}^\infty \frac{(\lambda (1 - p_2))^{n - x}}{(n - x)!} $$}
    \texttt{$$= e^{-\lambda} \frac{(\lambda p_2)^x}{x!} e^{\lambda (1 - p_2)} = \frac{(\lambda p_2)^x e^{-\lambda p_2}}{x!}
$$}
也就是说,尽管我们从泊松分布再加一个二项分布,得到的边缘分布仍然是一个\textbf{泊松分布}
\end{itemize}
\paragraph{似然函数}
设观测数据为 $ x_2 $,且 $ X_2 \sim \text{Poisson}(\lambda p_2) $,那么:

$$
P(X_2 = x_2) = \frac{(\lambda p_2)^{x_2} e^{-\lambda p_2}}{x_2!}
$$
\begin{itemize}
    \item 似然函数
    $$L(p_2, \lambda) = \prod_{i=1}^n \frac{(\lambda p_2)^{x_i} e^{-\lambda p_2}}{x_i!}$$
    \item 对数似然函数
    $$\ell(p_2, \lambda) = \sum_{i=1}^n \left[ x_i \log(\lambda p_2) - \lambda p_2 - \log(x_i!) \right]$$

    $$= \sum_{i=1}^n x_i \log(\lambda p_2) - n \lambda p_2 - \sum_{i=1}^n \log(x_i!)$$

\end{itemize}
\paragraph{最大化对数似然}
将 $ \lambda $ 和 $ p_2 $ 视为独立参数,分别对它们求偏导。
\begin{itemize}
    \item 对 $ \lambda $ 求偏导:    
$$
\frac{\partial \ell}{\partial \lambda} = \sum_{i=1}^n \frac{x_i}{\lambda p_2} - n p_2 = 0
$$

$$
\Rightarrow \frac{1}{\lambda p_2} \sum x_i = n p_2 \Rightarrow \frac{\bar{x}}{p_2} = n p_2 \Rightarrow \bar{x} = n p_2^2
$$
\item 对 $ p_2 $ 求偏导:

$$
\frac{\partial \ell}{\partial p_2} = \sum_{i=1}^n \frac{x_i}{p_2} - n \lambda = 0
\Rightarrow \frac{\bar{x}}{p_2} = n \lambda
\Rightarrow \bar{x} = n \lambda p_2
$$

\item 联立两个方程:

$ \bar{x} = n p_2^2 $ (来自 $ \partial/\partial \lambda $) 

$ \bar{x} = n \lambda p_2 $ (来自 $ \partial/\partial p_2 $)

$$
n \lambda p_2 = n p_2^2 \Rightarrow \lambda = p_2
$$

最大似然估计量:

$$
\hat{p}_2 = \sqrt{\frac{\bar{x}}{n}}, \quad \text{当 } \lambda = p_2
$$

\end{itemize}

\subsection{最大似然估计的无偏性与有效性}
\subsubsection{单独二项分布} 
$$
\hat{p}_1 = \frac{x_1}{n}
$$
\paragraph{无偏性}
$$
E\left[ \frac{X_1}{n} \right] = \frac{1}{n} E[X_1] = \frac{1}{n} (np_1) = p_1
$$
单独二项分布的最大似然估计是无偏的。
\paragraph{有效性}
\begin{itemize}
    \item 方差计算:
    $$
\text{Var}\left( \frac{X_1}{n} \right) = \frac{1}{n^2} \cdot \text{Var}(X_1) = \frac{1}{n^2} \cdot np_1(1 - p_1) = \frac{p_1(1 - p_1)}{n}
$$
    \item Fisher 信息量:
    对于 $ X \sim \text{Binomial}(n, p) $,Fisher 信息量为:

$$
I(p) = \frac{n}{p(1 - p)}
$$
    \item Cramér-Rao 下界: 
    $$
\frac{1}{n I(p)} = \frac{p(1 - p)}{n}
$$

与方差一致。
达到 Cramér-Rao 下界,独立二项分布的最大似然估计是有效的估计量。

\end{itemize}
\subsubsection{先泊松再二项}
$$
\hat{p}_2 = \sqrt{\frac{\bar{x}}{n}}, \quad \text{假设 } \lambda = p_2
$$
\paragraph{无偏性}
$$
E[\hat{p}_2] = E\left[ \sqrt{\frac{\bar{x}}{n}} \right]
$$

由于 $ \bar{x} \sim \text{Poisson}(\lambda p_2) $,且 $ \hat{p}_2 = \sqrt{\frac{\bar{x}}{n}} $ 是非线性变换,因此:

$$
E[\hat{p}_2] \neq \sqrt{ \frac{E[\bar{x}]}{n} } = \sqrt{ \frac{\lambda p_2}{n} }
$$

而真实值是 $ p_2 $,显然不等于上式。

结论:
先泊松再二项的最大似然估计不是无偏的。
\paragraph{有效性}
\begin{itemize}
    \item Cramér-Rao 下界
    对于参数 $ p_2 $,Fisher 信息量为:
$$
I(p_2) = \frac{n}{p_2(1 - p_2)}
$$

    \item 实际方差
    $$
\hat{p}_2^{\text{MLE}} = \sqrt{\frac{\bar{x}}{n}}, \quad \bar{x} \sim \text{Poisson}(\lambda p_2)
$$

用泰勒展开近似期望和方差:

$$
E[\hat{p}_2^{\text{MLE}}] \approx \sqrt{ \frac{\lambda p_2}{n} } + \frac{1}{2} \cdot \frac{1}{\sqrt{ \frac{\lambda p_2}{n} }} \cdot \frac{\lambda p_2}{n}
= \sqrt{ \frac{\lambda p_2}{n} } + \frac{1}{2} \sqrt{ \frac{\lambda p_2}{n} }
= \frac{3}{2} \sqrt{ \frac{\lambda p_2}{n} }
$$

$$
\text{Var}(\hat{p}_2^{\text{MLE}}) \approx \frac{1}{4} \cdot \frac{\text{Var}(\bar{x})}{E[\bar{x}]}
= \frac{1}{4} \cdot \frac{\lambda p_2}{\lambda p_2 / n} = \frac{n}{4}
$$

这与期望不符,说明估计量的方差也偏离了理论下界。
\end{itemize}
先泊松再二项的最大似然估计在一般情况下不是有效的。
\subsection{两种模型的差异比较}
\begin{itemize}
    \item 参数结构差异(固定试验 vs 随机试验次数):  
        单独二项分布中,试验次数 $ n $ 是固定的,参数为成功概率 $ p $;  
        而在“先泊松再二项”模型中,试验次数 $ N \sim \text{Poisson}(\lambda) $ 是随机的,因此需要同时估计两个参数 $ p $ 和 $ \lambda $。

    \item 期望与方差形式差异(如复合模型方差更大):  
        单独二项分布的期望为 $ E[X] = np $,方差为 $ \text{Var}(X) = np(1-p) $;  
        在“先泊松再二项”模型中,边缘分布为 $ X \sim \text{Poisson}(\lambda p) $,其期望为 $ E[X] = \lambda p $,方差为 $ \text{Var}(X) = \lambda p $。  
        因此,在相同期望下,复合模型的方差通常更大。

    \item 应用场景差异(稀有事件计数 vs 抽样调查):  
        单独二项分布适用于抽样调查、质量检测等固定试验次数的场景;  
        “先泊松再二项”模型更适合描述稀有事件或随机发生次数的场景,例如网络流量、顾客到达、罕见疾病病例等。
\end{itemize}

\end{document}
